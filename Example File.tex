\documentclass[12pt,letterpaper]{article}
\usepackage[margin=1in]{geometry}
\usepackage{MatrixExt}
\usepackage{verbatim}
\setlength{\parindent}{0cm}


\newcommand{\vertsp}{\vspace{0.4 in}}
\begin{document}

\section*{Linear Algebra Package}
The package is named \begin{verbatim} MatrixExt \end{verbatim} 
and can be used by simply typing \begin{verbatim} \usepackage{MatrixExt} \end{verbatim} 
in the preamble of a TeX document, provided that the .sty file is available.  The package requires the use of several packages.\footnote{It requires the following packages $\{$\emph{ifthen, amsmath}$\}$}
\vertsp

{\bf Never Indent:}
\begin{verbatim}

\neverindent
\end{verbatim}

\vertsp

{\bf To make "x" a bold vector:}
\begin{verbatim}

$\vec{x}$
\end{verbatim} 
Example: $\vec{x}$\\
\vertsp

{\bf To make a "C" in mathcal:}
\begin{verbatim}

$\mc{C}$
\end{verbatim}
Example: $\mc{C}$\\
\vertsp

{\bf To make a $n \times m$ matrix:}
\begin{verbatim}
$ \matrix{
\end{verbatim}
$a_1 \& a_2 \& \dots \& a_n \backslash \backslash$
$b_1 \& b_2 \& \dots \& b_n \backslash \backslash$
$\dots \backslash \backslash$
$m_1 \& m_2 \& \dots \& m_n$ \$
\vspace{10 pt}

\}

Example: $\matrix{a_1 & a_2 & \dots & a_n \\ b_1 & b_2 & \dots & b_n \\ \vdots & \vdots & & \vdots \\ m_1 & m_2 & \dots & m_n}$

\pagebreak


{\bf To make a $3 \times 3$ matrix augmented with an additional column:}
\begin{verbatim}
$\augmatrix{ccc|c}{
\end{verbatim}
$a \& b \& c \& d \backslash \backslash$
$e \& f \& g \& h \backslash \backslash$
$i \& j \& k \& l$ \$
\vspace{10 pt}

\}


Example: {\large{ $\augmatrix{ccc|c}{a & b & c & d \\ e & f & g & h \\ i & j & k & l}$}}
\vertsp


{\bf To get an iterative list from 1 to an upper value, you'll want:}
\begin{verbatim}
$\oneton[12]{\vec{u}}{+}$
\end{verbatim}
This gives: $\oneton[12]{\vec{u}}{+}$\\

You can also omit the upper bound (defaulting to $n$) and change the foldr connection:
\begin{verbatim}
$\oneton{\vec{v}}{,}$
\end{verbatim}
This gives: $\oneton{\vec{v}}{,}$

\vertsp

{\bf We have derivatives (partial and full) with optional commands for the numerator:}

\begin{verbatim}
$\deriv{x}$

$\deriv[f]{x}$

$\pderiv{t}$

$\pderiv[y]{x}$
\end{verbatim}
These give, respectively,
\begin{align*}
&\deriv{x} &
& \deriv[f]{x} &
\pderiv{t}&
& \pderiv[y]{x}&
\end{align*}
\vertsp


{\bf We have my personal favorite symbol for contradiction:}
\begin{verbatim}
$\contradiction$
\end{verbatim}
Which gives: $\contradiction$
\vertsp


{\bf And finally, we have inner products:}
\begin{verbatim}
$\ip{\vec{u}}{\vec{v}}$
\end{verbatim}
Which gives: $\ip{\vec{u}}{\vec{v}}$



\end{document}